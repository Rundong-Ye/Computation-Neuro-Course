\documentclass{article}
\usepackage{amsmath}
\usepackage{hyperref}
\title{Problem Set I}
\date{Due Wednesday, Oct  14, 2020}

\begin{document}
\maketitle

\section*{Note on the Homework}

Please read \href{https://github.com/Wenlab/Computation-Neuro-Course/wiki/%E4%BD%9C%E4%B8%9A%E6%8F%90%E4%BA%A4}{how to submit a homework} in our github page.
Please choose to solve 3 out of the following 4 problems. People who work out all 4 problems will have extra credits that will be weighted in the final score of the course. 

\section*{How granule cells sample inputs}

We have discussed about the synaptic organization of the cerebellum. Mossy fibers, which are long range projections from various brain regions, make connections with granule cells, the most numerous ( $ \sim 10^{11}$) neurons in the whole brain. These granule cells have a very small convergence: each of which only receives inputs from a handful mossy fibers. The outputs of granule cells, called parallel fibers, travel along the cerebellar cortex for a few millimeters and synapse onto Purkinje dendrites. Purkinje cells have the highest convergence in the brain, each of which receives inputs from more than $10^5$ synapses from parallel fibers. The granule-to-Purkinje projections are what we called in the class ”Connecting dense array to sparse array with extreme convergence and divergence”.
\\
\\
Why do we need so many granule cells? What can we say about the number of granule cells $N$, the number of mossy fiber inputs $M$, and the convergence of a granule cell $K$? Perhaps each granule cell is sampling a different combination of mossy fiber inputs. The higher the functional diversity, the more powerful computation downstream circuits (e.g., Purkinje dendrites) could perform, such as classification.
\\
\\
Assume each granule cell can choose $K$ inputs out of all $M$ mossy fibers, the number of possibilities is simply a binomial coefficient $\binom{M}{K}$. Now we ask the following questions.

\begin{itemize}
\item What is the probability $p$ that granule cells all receive different combinations of inputs?
\item For given $N$ and $M$, plot $p$ as a function of $K$, and show when $p$ reaches its maximum.
\item Using $N = 21000$, $M = 7000$, compute $K$ when $p$ approaches 95 percent of its maximum.
\item Discuss whether it is beneficial to have small $K$ when $M$ is very large.
\end{itemize}


\section*{Visualization of dendritic morphology}

Attached you will find morphology data txt files (.swc) of one pyramidal dendrite, one Purkinjie dendrite, and one arbor from larval zebrafish. Use MATLAB or Python to write a simple program: 
\begin{itemize}
\item Load the data file. 
\item Plot and visualize the neuronal 3D arbor shape. 
\item Calculate how many branching points on the dendritic arbors.
\item Perform a Sholl plot. Center on the cell body and draw spheres, and plot the number of intersections between sphere and dendrites as a function of sphere radius.  
\end{itemize}
In the swc file, each column has the following meaning (from left to right): segment index, segment type (cell body =1,, dendrite = 3), x coordinate ($\mu$m), y coordinate ($\mu$m), z coordinate ($\mu$m), segment diameter ($\mu$m), father segment index (root index = -1). 

\section*{Derivation of the Goldman-Hodgkin-Katz formula for reversal potential}

The reversal potential we discussed in the class only take into account one type of ion. However, some channels are not quite selective, and we need to combine the current flow from multiple ions, and the result is the Goldman-Hodgkin-Katz formula for membrane potential. I will write down the equation here, and it is your homework to provide the derivation of this formula.
\begin{equation} V_{m}=\frac{k_{B}T}{e}\ln\left(\frac{\sum_{i=1}^{N}P_{M_{i}^{+}}[M_{i}^{+}]_{out}+\sum_{j=1}^{N}P_{A_{j}^{-}}[A_{j}^{-}]_{in}}{\sum_{i=1}^{N}P_{M_{i}^{+}}[M_{i}^{+}]_{in}+\sum_{j=1}^{N}P_{A_{j}^{-}}[A_{j}^{-}]_{out}}\right).
\end{equation}
Here $P$ denotes the permeability of a given ion.


\section*{Integrate and Fire Neuron}

An integrate and fire neuron has a subthreshold membrane potential that
obeys the equation

\bigskip
\begin{equation}
C\frac{dV}{dt}=-\frac{V}{R}+I(t)  \label{eqn:IF}
\end{equation}
Once the voltage crosses threshold, the neuron fires a spike and $V$ is
reset to $0 $, as discussed in the class.
\\
\\
Consider the input current

\begin{equation}
I(t)=Q\sum_{k=-\infty }^{\infty }\delta (t-kT)
\end{equation}
where a charge Q crosses the membrane periodically.

\begin{itemize}
\item  Derive the subthreshold membrane potential as a function of time
for this input current. Describe the transient and steady state
behavior of the potential. Illustrate the result by qualitative or
(even better) numerically quantitative graphs.

\item Under what conditions will the neuron fire spikes? Compute the
firing frequency of the neuron when the conditions for firing are
met. Plot the firing frequency as a function of the interesting
variables and explain.\newline
\end{itemize}




\end{document}